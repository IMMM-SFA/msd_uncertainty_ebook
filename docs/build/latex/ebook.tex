%% Generated by Sphinx.
\def\sphinxdocclass{report}
\documentclass[letterpaper,10pt,english]{sphinxmanual}
\ifdefined\pdfpxdimen
   \let\sphinxpxdimen\pdfpxdimen\else\newdimen\sphinxpxdimen
\fi \sphinxpxdimen=.75bp\relax
\ifdefined\pdfimageresolution
    \pdfimageresolution= \numexpr \dimexpr1in\relax/\sphinxpxdimen\relax
\fi
%% let collapsable pdf bookmarks panel have high depth per default
\PassOptionsToPackage{bookmarksdepth=5}{hyperref}

\PassOptionsToPackage{warn}{textcomp}
\usepackage[utf8]{inputenc}
\ifdefined\DeclareUnicodeCharacter
% support both utf8 and utf8x syntaxes
  \ifdefined\DeclareUnicodeCharacterAsOptional
    \def\sphinxDUC#1{\DeclareUnicodeCharacter{"#1}}
  \else
    \let\sphinxDUC\DeclareUnicodeCharacter
  \fi
  \sphinxDUC{00A0}{\nobreakspace}
  \sphinxDUC{2500}{\sphinxunichar{2500}}
  \sphinxDUC{2502}{\sphinxunichar{2502}}
  \sphinxDUC{2514}{\sphinxunichar{2514}}
  \sphinxDUC{251C}{\sphinxunichar{251C}}
  \sphinxDUC{2572}{\textbackslash}
\fi
\usepackage{cmap}
\usepackage[T1]{fontenc}
\usepackage{amsmath,amssymb,amstext}
\usepackage{babel}



\usepackage{tgtermes}
\usepackage{tgheros}
\renewcommand{\ttdefault}{txtt}



\usepackage[Bjarne]{fncychap}
\usepackage{sphinx}

\fvset{fontsize=auto}
\usepackage{geometry}


% Include hyperref last.
\usepackage{hyperref}
% Fix anchor placement for figures with captions.
\usepackage{hypcap}% it must be loaded after hyperref.
% Set up styles of URL: it should be placed after hyperref.
\urlstyle{same}


\usepackage{sphinxmessages}
\setcounter{tocdepth}{1}



\title{ebook}
\date{May 27, 2021}
\release{v0.1.0}
\author{Battelle Memorial Institute}
\newcommand{\sphinxlogo}{\vbox{}}
\renewcommand{\releasename}{Release}
\makeindex
\begin{document}

\pagestyle{empty}
\sphinxmaketitle
\pagestyle{plain}
\sphinxtableofcontents
\pagestyle{normal}
\phantomsection\label{\detokenize{index::doc}}



\chapter{Testing out mathjax}
\label{\detokenize{index:testing-out-mathjax}}
\sphinxAtStartPar
The \sphinxstylestrong{first\sphinxhyphen{}order sensitivity index} indicates the percent of model output variance contributed by a factor individually (i.e., the effect of varying \sphinxstyleemphasis{x}$_{\text{i}}$alone) and is obtained using the following (Saltelli, 2002a; Sobol, 2001):
\begin{equation*}
\begin{split}S_i^1 = \frac{V_{x_i} [E_{x\sim_i} (x_i)]}{V(y)}\end{split}
\end{equation*}
\sphinxAtStartPar
with \sphinxstyleemphasis{E} and \sphinxstyleemphasis{V} denoting the expected value and the variance, respectively.


\chapter{Testing out a codeblock}
\label{\detokenize{index:testing-out-a-codeblock}}
\begin{sphinxVerbatim}[commandchars=\\\{\},numbers=left,firstnumber=1,stepnumber=1]
 \PYG{k+kn}{import} \PYG{n+nn}{ebook}

 \PYG{n}{ebook}\PYG{o}{.}\PYG{n}{plot\PYGZus{}experimental\PYGZus{}design}\PYG{p}{(}\PYG{p}{)}
\end{sphinxVerbatim}


\chapter{Testing out a note}
\label{\detokenize{index:testing-out-a-note}}
\begin{sphinxadmonition}{note}{Note:}
\sphinxAtStartPar
Keep track of the latest in programming solutions on the \sphinxhref{https://waterprogramming.wordpress.com/}{Water Programming} blog!
\end{sphinxadmonition}


\chapter{Chapters}
\label{\detokenize{index:chapters}}

\section{1.0 Introduction}
\label{\detokenize{1.0_Introduction:introduction}}\label{\detokenize{1.0_Introduction::doc}}
\sphinxAtStartPar
This guidance text has been developed in support of the Integrated Multisector Multiscale Modeling (IM3) Science Focus Area’s objective to formally integrate uncertainty into its research tasks. IM3 is focused on innovative modeling to explore how human and natural system landscapes in the United States co\sphinxhyphen{}evolve in response to short\sphinxhyphen{}term shocks and long\sphinxhyphen{}term influences. The project’s challenging scope is to advance our ability to study the interactions between energy, water, land, and urban systems, at scales ranging from local (\textasciitilde{}1km) to the contiguous United States, while consistently addressing influences such as population change, technology change, heat waves, and drought. Uncertainty and careful model\sphinxhyphen{}driven scientific insights are central to IM3’s key MultiSector Dynamics (MSD) science objectives shown below.

\sphinxAtStartPar
\sphinxstylestrong{IM3 key MSD science objectives include:}

\sphinxAtStartPar
\sphinxstyleemphasis{Develop flexible, open\sphinxhyphen{}source, and integrated modeling capabilities that capture the structure, dynamic behavior, and emergent properties of the multiscale interactions within and between human and natural systems.}

\sphinxAtStartPar
\sphinxstyleemphasis{Use these capabilities to study the evolution, vulnerability, and resilience of interacting human and natural systems and landscapes from local to continental scales, including their responses to the compounding effects of long\sphinxhyphen{}term influences and short\sphinxhyphen{}term shocks.}

\sphinxAtStartPar
\sphinxstyleemphasis{Understand the implications of uncertainty in data, observations, models, and model coupling approaches for projections of human\sphinxhyphen{}natural system dynamics.}


\chapter{Indices and tables}
\label{\detokenize{index:indices-and-tables}}\begin{itemize}
\item {} 
\sphinxAtStartPar
\DUrole{xref,std,std-ref}{genindex}

\item {} 
\sphinxAtStartPar
\DUrole{xref,std,std-ref}{modindex}

\item {} 
\sphinxAtStartPar
\DUrole{xref,std,std-ref}{search}

\end{itemize}



\renewcommand{\indexname}{Index}
\printindex
\end{document}